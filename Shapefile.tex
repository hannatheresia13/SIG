\section{Shapefile}
\subsection{Pengertian Shapefile}
Shapefile ArcView memiliki format data tersendiri yang disebut dengan shapefiles. 
Shapefiles adalah format data yang menyimpan lokasi geometrik dan informasi atribut dari suatu feature geografis. 
Pada umumnya kita hanya butuh satu file kerja seperti file Microsoft Worl dengan extension file *.doc, 
akan tetapi shapefile memiliki perbedaan, yaitu bahwa satu shapefile memiliki beberapa file yang saling berkaitan satu sama lainnya. Shapefile yang merupakan file untuk menyimpan data vector dalam bentuk arcview. Shapefile inilah yang kemudian akan diolah dan dianalisis dalam berbagai pekerjaan spasial dengan arcview. Saat ditampilkan dalam lembar view,shapefile masih berbentuk sebuah theme. Data vector yang tidak disimpan dalam sebuah shapefile akan menjadi objek grafik dalam lembar view. Objek grafik akan hilang apabila dimasukkan dalam shapefile. Sehingga, sebelum proses digitasi akan dilakukan, terlebih dahulu harus dibentuk sebuah shapefile kosong untuk wadah data vector yang telah terdigitasi.
. 


Beberapa file ini memiliki extension yang 42 berbeda-beda yang disimpan dalam workspace yang sama.


Catatan : tiga file extension pertama adalah bagian file extension yang harus ada dalam sebuah shapefile, file extension berikutnya sifatnya optional
